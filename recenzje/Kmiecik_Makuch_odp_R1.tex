% !TeX spellcheck = pl_PL
\documentclass[12pt]{article}
\usepackage[left=2.5cm, right=2.5cm]{geometry}

\usepackage[english,polish]{babel}

% Użyj polskiego łamania wyrazów (zamiast domyślnego angielskiego).
\usepackage{polski}
\usepackage[utf8]{inputenc}
\pagestyle{empty}

\parindent 10mm
\parskip 3mm

\begin{document}
    
    \begin{center}
        \textbf{Odpowiedź na recenzję }\\
        \vspace{10pt}
        Tytuł artykułu: \textit{Wybrane aspekty modyfikacji obudowy zestawu głośnikowego} \\
        Autorzy: \textit{Michał Kmiecik, Teresa Makuch}
    \end{center}

    Dziękujemy za pozytywną recenzję oraz cenne uwagi dotyczące pracy, do których odnosimy się poniżej.
    
    Wszystkie przedstawione wykresy charakterystyk kierunkowości zostały powiększone, dzięki czemu zyskały na czytelności.
    
    Rzeczywiście, w~toku pracy nacisk został przeniesiony na kwestię parametrów przetworników, mimo że pierwotny zamysł zakładał skupienie się na obudowie zestawu głośnikowego. Jednak zmiana tytułu nie była możliwa ze względu na wcześniejsze zgłoszenie pracy na konferencję.
    
    Karta katalogowa zamieszczona w~pracy jest jedyną udostępnioną przez producenta dla tego modelu głośnika, mimo że jest dostępny w~dwóch wersjach impedancji. Autorzy zdają sobie sprawę, że nominalna wartość impedancji badanych przetworników jest inna, niż przetwornika, który opisuje karta, co zostało podkreślone w~pracy. Ponadto, jak podaje producent, parametry podane w~karcie zostały zmierzone po uprzednim obciążeniu głośnika i~taki był cel jej umieszczenia w~pracy: wskazanie zmienności parametrów pod wpływem zużycia przetwornika. Wartości zmierzone na końcowym etapie są bliższe tym zawartym w~karcie katalogowej. Należy również zauważyć, że w~procesie projektowania zestawu głośnikowego karta katalogowa jest podstawowym źródłem informacji o~przetworniku, pozwalającym na podjęcie decyzji o~wyborze modelu. Takie uzasadnienie jej umieszczenia w~zestawieniu dodano do artykułu.
    
\end{document}