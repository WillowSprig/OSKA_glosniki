% !TeX spellcheck = pl_PL
\documentclass[12pt]{article}
\usepackage[left=2.5cm, right=2.5cm]{geometry}

\usepackage[english,polish]{babel}

% Użyj polskiego łamania wyrazów (zamiast domyślnego angielskiego).
\usepackage{polski}
\usepackage[utf8]{inputenc}
\pagestyle{empty}
\usepackage{indentfirst}

\usepackage{siunitx}

\parindent 10mm
\parskip 3mm

\begin{document}
    
    \begin{center}
        \textbf{Odpowiedź na recenzję }\\
        \vspace{10pt}
        Tytuł artykułu: \textbf{Wybrane aspekty modyfikacji obudowy zestawu głośnikowego} \\
        Autorzy: \textbf{Michał Kmiecik, Teresa Makuch}
    \end{center}
    

    Dziękujemy za pozytywną recenzję pracy, komentarze dotyczące sfery merytorycznej oraz liczne wskazówki edycyjne; pragniemy ustosunkować się do przekazanych nam uwag.
    
    Edytowano rozdziały 5.1 -- 5.3, rozwijając te opisy, które były zbyt krótkie i~lapidarne oraz dodając odwołania do odpowiednich rysunków i~tabel.
    
    Większość uwag zamieszczonych w~treści została uwzględniona, w~sposób oczywisty zwiększając komunikatywność tekstu. Poniżej odniesienie do uwag, do których Autorzy nie zdecydowali się zastosować i~uzasadnienie tych decyzji (przytoczono fragmenty artykułu, których te uwagi dotyczyły).
    
    \subsubsection*{Abstract}
    
    Wprowadzono zmiany, które nie wpłynęły na zgodność treści angielskiej i~polskiej wersji abstraktu. Termin \textit{directivity characteristics} został użyty m.in. za L.~Berankiem (\textquotedblleft Acoustics: Sound Fields and Transducers\textquotedblright, Oxford 2012), M.~Kleinerem (\textquotedblleft Electroacoustics\textquotedblright, 2013) i~D.~Davisem (\textquotedblleft Sound System Engineering\textquotedblright, Oxford 2006).
    
    \subsubsection*{1.1. Cel pracy}
    
    \textit{Prace mają na celu udoskonalenie konstrukcji obudowy zestawu głosnikowego} [\dots]. \textit{Pożądane jest osiągnięcie maksymalnie
    płaskiej charakterystyki amplitudowo-częstotliwościowej} [\dots].
    
    Użyto czasu teraźniejszego ze względu na ciągłość prac nad zagadnieniem.
    
    \subsubsection*{4.1. Wprowadzenie}
    
    \textit{Dzięki stworzeniu modelu komputerowego możliwe jest łatwe i~niewymagające wysokich kosztów testowanie różnych rozwiązań metodami numerycznymi.}
    
    Zdaniem autorów sformułowanie „tworzenie modelu komputerowego” trafniej oddaje specyfikę tego procesu, niż jego „opracowywanie”.
    
    \subsubsection*{5.1. Pomiary impedancji}
    
    %Co z tymi etapami?
    \textit{Porównano także wyniki pomiaru metodą dodanej masy dla tego samego głośnika po pierwszym etapie [\dots].}
    
    Opis poszczególnych etapów zużycia przeniesiono z~części~3.1. do części~3., gdzie jest on bardziej uwydatniony; dodano także odnośnik do tego opisu w~zacytowanym powyżej fragmencie, aby ułatwić czytelnikowi orientację w~tekście. Autorzy uznali, że użycie opisów „po obciążeniu mocą \SI{40}{\watt} przez \SI{2}{\hour}” oraz „po obciążeniu mocą \SI{60}{\watt} przez \SI{10}{\minute}” nie wpłynęłoby korzystnie na czytelność i~komunikatywność tekstu, a~tylko takie opisy charakteryzują w~pełni każdy z~etapów.
    
    \textit{Na wykresach dla dwóch głośników w~obudowie z~otworami widoczne są dwa rezonanse: pierwszy z~nich to rezonans powietrza w obudowie, charakterystyczny dla obudowy typu \emph{bass reflex}; drugi to rezonans głośnika, przesunięty w kierunku wyższych częstotliwości}
    
    Liczebniki „pierwszy” i~„drugi” zostały użyte podobnie, jak w~przypadku składowych częstotliwościowych, numerowanych rosnąco wraz ze wzrostem częstotliwości.
    
    \subsubsection*{5.4. Drgania obudowy}
    
    \textit{Na widmie przemieszczeń (rysunek 16) widoczne jest lokalne maksimum amplitudy dla częstotliwości 400~Hz, co jednak nie znajduje odzwierciedlenia w widmie poziomu ciśnienia akustycznego}
    
    Opis dotyczy widm poziomu ciśnienia akustycznego ze wszystkich 5~punktów -- dodano odwołania do odpowiednich rysunków.
    
    \subsubsection*{6. Podsumowanie}
    
    \textit{Otrzymane dotychczas wyniki wskazują, iż optymalnym byłoby przyjęcie częstotliwości granicznych dla głośnika wysoko- i~niskotonowego w~przedziale \num{1000}~--~\SI{1200}{\hertz}.}
    
    W~treści artykułu dodano argumentację uzasadniającą taki wybór punktu podziału.
    
    
\end{document}